% !TeX document-id = {32a890c7-0db2-461b-b834-605db552d4de}
% !TEX root
% !TEX program = xelatex
% !BIB program = biber

%\def \PrintMode{} %在使用电子版论文时,请将此行注释。在打印纸质论文时,请保持本行命令不被注释,然后打印时选择双面打印即可。

%用来控制是否启动打印模式的宏,请勿改动。
\ifx \PrintMode \undefined
	\def \SideMode{oneside}
	\def \ClearPageStyle{\clearpage}
\else
	\def \SideMode{twoside}
	\def \ClearPageStyle{\cleardoublepage}
\fi

\usepackage[heading,zihao=-4]{ctex} %用来提供中文支持

\usepackage{xspace} %提供一些好用的空格命令

\usepackage{hyperref}
\hypersetup{%
	%  dvipdfmx,% 设定要使用的 driver 为 dvipdfmx
	unicode={true},% 使用 unicode 来编码 PDF 字符串
	pdfstartview={FitH},% 文档初始视图为匹配宽度
	bookmarksnumbered={true},% 书签附上章节编号
	bookmarksopen={true},% 展开书签
	pdfborder={0 0 0},% 链接无框
	citecolor=blue,
	linkcolor=blue, % blue
	anchorcolor=green,
	urlcolor=blue,
	colorlinks=true,     %注释掉此项则交叉引用为彩色边框(将colorlinks和pdfborder同时注释掉)
	pdfborder=000        %注释掉此项则交叉引用为彩色边框
	%pdfstartview=FitH,
	%pdfpagemode=FullScreen % 实现打开后全屏
}

\usepackage{cleveref} %交叉引用


%使用biblatex管理文献,输出格式使用gb7714-2015标准,后端为biber
\usepackage[backend=biber,style=gb7714-2015,hyperref=true,gbnamefmt = lowercase,gbnamefmt=familyahead, giveninits = true,maxbibnames=99]{biblatex}

%屏蔽无关的Warning
\usepackage{silence}
\WarningFilter*{biblatex}{Conflicting options.\MessageBreak'eventdate=iso' requires 'seconds=true'.\MessageBreak Setting 'seconds=true'}



%============================版面控制宏包=================================%
% 页边距:上:3.0cm,下:2.0cm,左:2.8cm,右:2.2cm,页眉:2.2cm,页脚1.5cm;
\usepackage{geometry}
\geometry{top=1.8cm,bottom=2cm,left=2.8cm,right=2.2cm,includehead,includefoot}
\usepackage[symbol,perpage]{footmisc}  % 脚注控制可使得每页的脚注编码重新复位,

\usepackage{fontspec} %设置字体需要的宏包
%设置西文字体为Times New Roman,如果没有则以开源近似字体代替
\IfFontExistsTF{Times New Roman}{
	\setmainfont{Times New Roman}
}{
	\usepackage{newtxtext,newtxmath}
}

%设置文档中文字体。优先次序:中易 > Adobe > 华文(Mac) > Fandol
\IfFontExistsTF{SimSun}{
	\setCJKmainfont[AutoFakeBold=2,ItalicFont=KaiTi]{SimSun}
}{
	\IfFontExistsTF{AdobeSongStd-Light}{
		\setCJKmainfont[AutoFakeBold=2,ItalicFont=AdobeKaitiStd-Regular]{AdobeSongStd-Light}
	}{
		\IfFontExistsTF{STSong}{
			\setCJKmainfont[AutoFakeBold=2,BoldFont=STHeiti,ItalicFont=STKaiti]{STSong}
		}{
			\setCJKmainfont[AutoFakeBold=2,BoldFont=FandolHei-Regular,ItalicFont=FandolKai-Regular]{FandolSong-Regular}
		}
	}
}
\IfFontExistsTF{SimHei}{
	\setCJKsansfont[AutoFakeBold=2]{SimHei}
}{
	\IfFontExistsTF{AdobeHeitiStd-Regular}{
		\setCJKsansfont[AutoFakeBold=2]{AdobeHeitiStd-Regular}
	}{
		\IfFontExistsTF{STHeiti}{
			\setCJKsansfont [AutoFakeBold=2]{STHeiti}
		}{
			\setCJKsansfont[AutoFakeBold=2]{FandolHei-Regular}
		}
	}
}

\renewcommand{\baselinestretch}{1.5} %1.5倍行距


\showboxdepth=5
\showboxbreadth=5


%=============================标题与列表宏包=============================%
%\usepackage{titlesec}  #             % 控制标题的宏包,配合命令在后面,
\usepackage{enumerate}              % 改变列表标号样式宏包 其后可接选项[a,A,i,I,1]
\usepackage[font=small]{caption}               % 浮动图形和表格标题样式,可选项为
% [scriptsize,footnotesize,centerlast]
\usepackage{setspace}               % 图形和表格的标题如果是多行,行距比较大,可以加宏包
%使用绝对坐标制作封面使用的宏包
\usepackage[absolute,overlay]{textpos}
\setlength{\TPHorizModule}{1mm}
\setlength{\TPVertModule}{1mm}

\usepackage{fancyvrb,shortvrb,fancyhdr}  % 支持抄录
% \usepackage{cite} % natbib不能再用了

\lhead{{\CompleteYear}年~华东师范大学学士学位论文}
\chead{}
\rhead{\TitleCHS}
\lfoot{}
\cfoot{\thepage}
\rfoot{}


%======================== 数学公式图形表格相关宏包 ===============================%
\usepackage{xcolor}
\usepackage{amsmath,amsthm,amssymb,latexsym}
\newcommand\hmmax{0} % default 3
\newcommand\bmmax{0} % default 4
\usepackage{bm}
\usepackage{mathrsfs}                % 不同于\mathcal or \mathfrak 之类的英文花体字体
\usepackage{subeqnarray}             %多个子方程(1-1a)(1-1b)
\usepackage[subnum]{cases}           % 公式环境cases
\usepackage{array,tabularx,longtable,booktabs,threeparttable,colortbl,multirow,bigstrut}
\usepackage{dcolumn}            % 让表格中将小数点对齐
\usepackage{graphicx,subfigure}
\usepackage{flafter}       % 因为图形可浮动到当前页的顶部,所以它可能会出现
% 在它所在文本的前面. 要防止这种情况,可使用 flafter宏包
\usepackage[below]{placeins}    %浮动图形控制宏包

%=========================== 特殊文本元素宏包 ==============================%
\usepackage{nicefrac}                % 在正文文本中排版分式时,可以用它来得到较好的排版效果。
\usepackage{units}                   % 基于 nicefrac 宏包,提供对计量单位比较美观的排版效果。
\usepackage{listings}                % 源代码宏包

\usepackage{makeidx}                 % 建立索引宏包
\makeindex

\pagestyle{fancy}
\setlength{\headheight}{20pt}
\hbadness=10000
\tolerance=10000
